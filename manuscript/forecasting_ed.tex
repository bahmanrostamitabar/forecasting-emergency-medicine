% Options for packages loaded elsewhere
\PassOptionsToPackage{unicode}{hyperref}
\PassOptionsToPackage{hyphens}{url}
\PassOptionsToPackage{dvipsnames,svgnames,x11names}{xcolor}
%
\documentclass[
  authoryear,
  preprint,
  3p]{elsarticle}

\usepackage{amsmath,amssymb}
\usepackage{lmodern}
\usepackage{iftex}
\ifPDFTeX
  \usepackage[T1]{fontenc}
  \usepackage[utf8]{inputenc}
  \usepackage{textcomp} % provide euro and other symbols
\else % if luatex or xetex
  \usepackage{unicode-math}
  \defaultfontfeatures{Scale=MatchLowercase}
  \defaultfontfeatures[\rmfamily]{Ligatures=TeX,Scale=1}
\fi
% Use upquote if available, for straight quotes in verbatim environments
\IfFileExists{upquote.sty}{\usepackage{upquote}}{}
\IfFileExists{microtype.sty}{% use microtype if available
  \usepackage[]{microtype}
  \UseMicrotypeSet[protrusion]{basicmath} % disable protrusion for tt fonts
}{}
\makeatletter
\@ifundefined{KOMAClassName}{% if non-KOMA class
  \IfFileExists{parskip.sty}{%
    \usepackage{parskip}
  }{% else
    \setlength{\parindent}{0pt}
    \setlength{\parskip}{6pt plus 2pt minus 1pt}}
}{% if KOMA class
  \KOMAoptions{parskip=half}}
\makeatother
\usepackage{xcolor}
\setlength{\emergencystretch}{3em} % prevent overfull lines
\setcounter{secnumdepth}{5}
% Make \paragraph and \subparagraph free-standing
\ifx\paragraph\undefined\else
  \let\oldparagraph\paragraph
  \renewcommand{\paragraph}[1]{\oldparagraph{#1}\mbox{}}
\fi
\ifx\subparagraph\undefined\else
  \let\oldsubparagraph\subparagraph
  \renewcommand{\subparagraph}[1]{\oldsubparagraph{#1}\mbox{}}
\fi


\providecommand{\tightlist}{%
  \setlength{\itemsep}{0pt}\setlength{\parskip}{0pt}}\usepackage{longtable,booktabs,array}
\usepackage{calc} % for calculating minipage widths
% Correct order of tables after \paragraph or \subparagraph
\usepackage{etoolbox}
\makeatletter
\patchcmd\longtable{\par}{\if@noskipsec\mbox{}\fi\par}{}{}
\makeatother
% Allow footnotes in longtable head/foot
\IfFileExists{footnotehyper.sty}{\usepackage{footnotehyper}}{\usepackage{footnote}}
\makesavenoteenv{longtable}
\usepackage{graphicx}
\makeatletter
\def\maxwidth{\ifdim\Gin@nat@width>\linewidth\linewidth\else\Gin@nat@width\fi}
\def\maxheight{\ifdim\Gin@nat@height>\textheight\textheight\else\Gin@nat@height\fi}
\makeatother
% Scale images if necessary, so that they will not overflow the page
% margins by default, and it is still possible to overwrite the defaults
% using explicit options in \includegraphics[width, height, ...]{}
\setkeys{Gin}{width=\maxwidth,height=\maxheight,keepaspectratio}
% Set default figure placement to htbp
\makeatletter
\def\fps@figure{htbp}
\makeatother

\usepackage{booktabs}
\usepackage{longtable}
\usepackage{array}
\usepackage{multirow}
\usepackage{wrapfig}
\usepackage{float}
\usepackage{colortbl}
\usepackage{pdflscape}
\usepackage{tabu}
\usepackage{threeparttable}
\usepackage{threeparttablex}
\usepackage[normalem]{ulem}
\usepackage{makecell}
\usepackage{xcolor}
\makeatletter
\makeatother
\makeatletter
\makeatother
\makeatletter
\@ifpackageloaded{caption}{}{\usepackage{caption}}
\AtBeginDocument{%
\ifdefined\contentsname
  \renewcommand*\contentsname{Table of contents}
\else
  \newcommand\contentsname{Table of contents}
\fi
\ifdefined\listfigurename
  \renewcommand*\listfigurename{List of Figures}
\else
  \newcommand\listfigurename{List of Figures}
\fi
\ifdefined\listtablename
  \renewcommand*\listtablename{List of Tables}
\else
  \newcommand\listtablename{List of Tables}
\fi
\ifdefined\figurename
  \renewcommand*\figurename{Figure}
\else
  \newcommand\figurename{Figure}
\fi
\ifdefined\tablename
  \renewcommand*\tablename{Table}
\else
  \newcommand\tablename{Table}
\fi
}
\@ifpackageloaded{float}{}{\usepackage{float}}
\floatstyle{ruled}
\@ifundefined{c@chapter}{\newfloat{codelisting}{h}{lop}}{\newfloat{codelisting}{h}{lop}[chapter]}
\floatname{codelisting}{Listing}
\newcommand*\listoflistings{\listof{codelisting}{List of Listings}}
\makeatother
\makeatletter
\@ifpackageloaded{caption}{}{\usepackage{caption}}
\@ifpackageloaded{subcaption}{}{\usepackage{subcaption}}
\makeatother
\makeatletter
\@ifpackageloaded{tcolorbox}{}{\usepackage[many]{tcolorbox}}
\makeatother
\makeatletter
\@ifundefined{shadecolor}{\definecolor{shadecolor}{rgb}{.97, .97, .97}}
\makeatother
\makeatletter
\makeatother
\journal{Annals of Emergency Medicine}
\ifLuaTeX
  \usepackage{selnolig}  % disable illegal ligatures
\fi
\usepackage[]{natbib}
\bibliographystyle{elsarticle-harv}
\IfFileExists{bookmark.sty}{\usepackage{bookmark}}{\usepackage{hyperref}}
\IfFileExists{xurl.sty}{\usepackage{xurl}}{} % add URL line breaks if available
\urlstyle{same} % disable monospaced font for URLs
\hypersetup{
  pdftitle={Forecasting},
  pdfauthor={Bahman Rostami-Tabar; Rob J. Hyndman},
  pdfkeywords={Forecasting, Ambulance, Hierarchical structure},
  colorlinks=true,
  linkcolor={blue},
  filecolor={Maroon},
  citecolor={Blue},
  urlcolor={Blue},
  pdfcreator={LaTeX via pandoc}}

\setlength{\parindent}{6pt}
\begin{document}

\begin{frontmatter}
\title{Forecasting}
\author[1]{Bahman Rostami-Tabar%
\corref{cor1}%
\fnref{fn1}}
 \ead{rostami-tabarb@cardiff.ac.uk} 
\author[2]{Rob J. Hyndman%
%
}
 \ead{rob.hymdman@monash.ac.edu} 

\affiliation[1]{organization={Cardiff University, Cardiff Business
School},country={United Kingdom},countrysep={,},postcode={CF10
3EU},postcodesep={}}
\affiliation[2]{organization={Monash University, Department of
Econometrics and Business
Statistics},country={Australia},countrysep={,},postcode={VIC
3800},postcodesep={}}

\cortext[cor1]{Corresponding author}
\fntext[fn1]{This is the first author footnote.}

        
\begin{abstract}
This is the abstract. An accurate demand forecasting is crucial in
Emergency medicine to depict various courses of action that can result
in massive savings in terms of patient lives. Inability to match the
staff with the demand results in an overcrowding care system which is a
serious problem causing challenging situations on patient flow. Also, it
is related with increasing length of stay, low patient satisfaction,
increasing health care costs, inaccuracy in electronic medical record,
and reported waiting times.
\end{abstract}





\begin{keyword}
    Forecasting \sep Ambulance \sep 
    Hierarchical structure
\end{keyword}
\end{frontmatter}
    \ifdefined\Shaded\renewenvironment{Shaded}{\begin{tcolorbox}[frame hidden, interior hidden, breakable, enhanced, boxrule=0pt, borderline west={3pt}{0pt}{shadecolor}, sharp corners]}{\end{tcolorbox}}\fi

\hypertarget{introduction}{%
\section{Introduction}\label{introduction}}

An accurate demand forecasting is crucial in Emergency medicine to
depict various courses of action that can result in massive savings in
terms of patient lives. Inability to match the staff with the demand
results in an overcrowding care system which is a serious problem
causing challenging situations on patient flow. Also, it is related with
increasing length of stay, low patient satisfaction, increasing health
care costs, inaccuracy in electronic medical record, and reported
waiting times.without incurring last-minute expenses, such as overtime
or supplemental staffing {[}Reference{]}.

An accurate forecasting of the daily demand enables managers to match
staff to meet anticipated patients, reconfigure units and redeploy staff
and vehicles. This will have many advantages for patients, staff and the
quality of healthcare services provided.

The ability to accurately forecast the demand of future health services
is critical for planning decisions aimed at improving the quality of
service delivery in the health care. Forecasting has been used in
various fields of healthcare to inform planning decisions at the
strategic, tactical and operational levels. There is at least one or
multiple reasons why we need to produce a forecast. Most often reasons
determine clearly what to forecast and which level of forecast
granularities are required. There are typically some planning process at
strategic(long-term) level such as dimensioning resource capacities and
budgeting; tactical(medium-term) level such as temporary capacity
expansions, staff-shift scheduling, and inventory control; and
operational(short-term) level such as planning rosters for staff and
allocate resources and medical item distribution. Each planning level
will typically have different forecasting needs in terms of granularity
and horizon (how far into the future do we forecast). A natural question
that arise from these examples is how to generate forecasts to support
planning decisions at multiple granularities and horizons. In order to
answer this question, we first need to understand some fundamental
aspects of forecasting in healthcare:

Daily forecasts are required to inform the short-term planning for the
current and the upcoming shifts of the day. This involves the decision
making related to the execution of the delivery process for various
health care services such as Ambulatory ,Emergency. Daily forecasts are
important at various levels:

\begin{itemize}
\item
  \textbf{Nature of incidents}
\item
  \textbf{Category: Red, Amber, Green}
\item
  \textbf{Health board}
\item
  \textbf{Country level}
\end{itemize}

Implementing and sustaining improvements in hospital-wide flow requires
alignment, cooperation, and coordination between hospital units and
departments. Without effective executive oversight and collaboration,
teams operate in isolation from one another and the aggregated impact of
their efforts is limited. In many cases, this isolation leads to
duplicative work, rework, or work that runs counter to overall goals to
improve hospital-wide patient flow

The combination of demand forecast, incidents being attended, resource
availability and delays at hospitals, provide information on the state
of the unscheduled care system across the emergency medicine services.
Having this full picture enables the delivery managers to focus on the
areas that require intervention to enable the most effective delivery of
the service to the patients.

Despite a large number of studies dedicated to forecasting in
healthcare{[}ref{]}, this ques- tion unanswered in healthcare
forecasting. Almost all studies assume that the level of data
granularity matches with the level of forecast requirement, e.g.~daily
series is used to generate daily forecast. Moreover, in the presence of
the high data granularity, the recommendation is to first aggregate the
data into a requested forecast granularity level and then generate the
forecast(Goodwin, 2018). However, there is no empirical evidence to
support this claim

\hypertarget{lit}{%
\section{Research background}\label{lit}}

Table Table~\ref{tbl-lit} summarise studies on forecasting in emergency
and urgent care.

\hypertarget{tbl-lit}{}
\begin{table}[!h]
\caption{\label{tbl-lit}summary of literature review }\tabularnewline

\centering
\resizebox{\linewidth}{!}{
\begin{tabular}{l|r|l|l|>{\raggedright\arraybackslash}p{15em}|l}
\hline
Author & Year & Forecast variable & Forecast Horizon & Method & Measure\\
\hline
McCarthy 
2008 & 2008 & ED arrivals & 1 day & Poisson log-linear regression model, including temporal factors, patient characteristics and climatic factors & 95\% CI\\
\hline
Cote et al. & 2013 & ED arrivals & 1 day & Fourier regression & R\textasciicircum{}2,  Standard Error\\
\hline
Kim et al. & 2014 & Hospital demand & 1h, 7days, 
30days & Linear regression; Exponential smoothing; ARIMA; GARCH; VAR & MAPE\\
\hline
1 day & 2007 & Ambulance demand & 1day & Regression & RMSE\\
\hline
Hertzum & 2017 & ED arrivals
ED occupancy & 1day & linear regression; SARIMA; Naïve & MAE, MAPE, 
MASE\\
\hline
Choudhury and Urena & 2020 & ED arrivals & 1day & ARIMA; Holt-winters; TBATS; ANN & RMSE, ME\\
\hline
Steins et al. & 2019 & Ambulance call & 1day & ZIP and ZINB regression;
 moving average with seasonality weights & ME, MAE, RMSE\\
\hline
Jones et al. & 2009 & ED census & 1day & VAR;  Holt winters & MAE\\
\hline
Morzuch and Allen & 2006 & ED arrivals & 7 days & Regression; ARIMA; Exponential smoothing & RMSE\\
\hline
Taylor & 2008 & centre volume & 14 days & ARMA, Exponential smoothing; Dynamic harmonic regression; Seasonal random walk; Seasonal mean;
Random walk; Mean; Median; Simple exponential smoothing & MAE
RMSE\\
\hline
Gijo and Balakrishna & 2016 & call volume & 7 days & SARIMA & MMSE\\
\hline
\end{tabular}}
\end{table}

Focus on the ambulance services!

\citet{gijo2016sarima} generated a time series model to forecast the
daily and hourly call volume at all centre handling emergency ambulance
services. Since historical data showed seasonality, SARIMA models were
investigated. Regarding the daily model, the authors generated a SARIMA
model, which, however, resulted in the forecast error (MMSE) that
significantly increased when the lead time exceeded 8 days. On the other
hand, the SARIMA model proposed to forecast the log-calls an hourly
basis. This model was found to fit well the model both for shorter and
longer lead times.

Forecasting at the daily level of granularity is generally use for
roaster planing and deciding when to contact staff on call for instance.
\citet{luo2017hospital} use a combination of seasonal ARIMA (SARIMA) and
a single exponential smoothing (SES) model to forecast daily outpatient
visits 1 week ahead.They indicate that using combinatorial model can be
more effective than each model separately. \citet{whitt2019forecasting}
examine several alternative models to forecast the total daily arrival
for 1-7 days ahead, including a linear regression based on calendar and
weather variables, seasonal autoregressive integrated moving average
with exogenous regressors (SARIMAX) and the multilayer perceptron (MLP)
model. Using a daily ED admission for 3 years, they show that SARIMAX
provides the most accurate daily forecast by MSE measure.
\citet{marcilio2013forecasting} compare the forecast accuracy of the
Generalised Linear Model (GLM), Generalised Estimating Equations (GEE),
and Seasonal Autoregressive Integrated Moving Average (SARIMA) methods
using total daily patient visits to an ED using MAPE. They conclud that
GLM and GEE models provide more accurate forecasts than SARIMA model.
\citet{rostami2020anticipating} propose a model to forecast daily ED
attendance with consideration of different types of holidays, weekday
effects, auto-regressive effects, long-term trends and date effects.
They provide probabilistic forecasts to quantify uncertainties in future
ED attendance and they show that the proposed model outperforms three
time series techniques including 1) Naive, 2) AutoRegressive, AR(p), 3)
exponential smoothing state space model (ETS) and a regression model
without considering special events as alternatives. \citet{zhou2018time}
SARIMA, NARNN and the hybrid SARIMA-NARNN to forecast the daily number
of new admission inpatients. The root mean square error (RMSE), mean
absolute error (MAE) and mean absolute percentage error (MAPE) were used
to compare the forecasting performance among the three models. They show
that NARNN model outperforms the others. \citet{sun2009forecasting} use
SARIMA model to forecast daily patient volume for each patient acuity
level. MAPE is used to choose the best-fit model. They fitted separate
ARIMA models to the three categories of acuity and overall data. They
conclude that the ARIMA model is effective for both short term (weekly)
and long term (three months) forecast horizons. Moreover, they observe
that the impact of weather is not significant.
\citet{zinouri2018modelling} use SARIMA to develop a statistical
prediction model that provides short-term forecasts of daily surgical
demand. Our results suggest that the proposed SARIMA model can be useful
for estimating surgical case volumes 2--4 weeks prior to the day of
surgery, which can support robust and reliable staff schedules.
\citet{moustris2012seven} use Artificial Neural Network (ANN) models to
forecast the total weekly(7 days) number of Childhood Asthma Admission.
Three different ANN models were developed and trained to forecast the
admission of different age subgroups and for the whole study population.
\citet{mccoy2018assessment} compare SARIMA, Prophet and Snaive methods
for prediction 365 days of hospital discharge using a daily hospital
discharge volumes at 2 large, New England academic medical centers. They
show that Prophet model outperforms SARIMA and Snaive.

\citet{khaldi2019forecasting} investigate the combination of the
Artificial Neural Networks (ANNs) with a signal decomposition technique
named Ensemble Empirical Mode decomposition (EEMD), to make one step
ahead weekly forecasting of patients arrivals to ED. using seven years
of weekly demand. The results of the proposed model were compared
against ANN without signal decomposition, ANN with Discrete wavelet
Transform (DWT). The results show that the combined forecasts outperform
the benchmarking models.

\citet{steins2019forecasting} aimed to develop a forecasting model for
predicting the number of ambulance services calls per hour and
geographical areas, to support managers in decisions-making and to
investigate which ones were the factors that affected the number of
calls. Data collected consisted of a time and location of historical
ambulance call data for three counties in Sweden and a list of
explanatory factors of the area, such as socioeconomic and geographic.
In order to deal with large number of zeros in the data, authors
developed zero-inflated Poisson (ZIP) and zero-inflated negative
binomial (ZINB) regression models. These were then compared to the
currently existing forecasting system, based on moving average with
seasonality weights, using ME, MAE and RMSE. Firstly, the factors
affecting the number of ambulance calls were found to be the following:
population in different age groups, median income, length of road,
number of nightlife spots (the number of restaurants), day of the week
and hour of the day. Secondly, it was found that the older population
(65-100) generated more ambulance calls and that ZIP model performs
better than the current model. However, the improvement provided by the
more advance model was not much greater than the one provided by the
existing model. Authors suggested that it could be because the
population, which was used as an independent variable in both models,
was so dominant compared to the other factors. Moreover, both models
either underestimated or overestimated the number of calls. Authors
suggested that the inability to capture a positive trend resulted in
underestimation, and the opposite was due to a negative trend.
Therefore, authors recommended that further research should add certain
temporal variables able to capture the trend.

\hypertarget{data}{%
\section{Data}\label{data}}

Operations Department and I suppose it's split into three different
categories. You've got\ldots{} you've got Response which is the
ambulances. You've got the Control which is the people who take the
telephone calls. And then you've got Resilience which are the things
that people like have the hazardous response teams so they'll go and
save people from water and maybe with fires as well. They've got\ldots{}
they're specially trained in different things so there's the contingency
sort of there. So from a Control perspective, they look at\ldots{} they
look at it at least twice a week to see what the\ldots{} what it\ldots{}
how busy it is. And then they staff\ldots{} they\ldots{} they're looking
to staff to the worst level, knowing that they will get some
people\ldots{} there will be some abstractions. So they won't manage
to\ldots{} to manage that, because their key area {[}inaudible{]}.
Excuse me, they try to answer within the\ldots{} the phone within three
seconds 95\% of the time. I'm not sure exactly. It's something like
that. That's very high up there. That's what they're trying to do.
So\ldots{} and because I got it broken down at the Control level, so
we've got North Wales level for calls, and we've also got it broken down
by the North, the Mid, and the South Control. It's actually called the
North, the South East, and Central and West. Those are the three control
areas. So they can see how those break down, knowing that at the moment
we've got particularly high demand in the South East, and it's forecast
to be just bonkers

\hypertarget{design}{%
\section{Experimental set}\label{design}}

\hypertarget{result}{%
\section{Result and discussion}\label{result}}

\hypertarget{conclusion}{%
\section{Conclusion}\label{conclusion}}


\renewcommand\refname{References}
  \bibliography{mybibfile.bib}


\end{document}
